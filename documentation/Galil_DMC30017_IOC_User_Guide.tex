\documentclass[openany]{article}
\usepackage[a4paper,margin=1in,bottom=1.5in]{geometry} % define margins. Bottom margin is used to lift a little bit the page number.
\usepackage[english]{babel} % document language is english
\usepackage{tikz} % for drawing (currently not used).
\usepackage{graphicx} % for including images
\usepackage[export]{adjustbox}
\usepackage{fancyhdr} % used for creating headers and footers. only used in title page in this document.
\usepackage{tabularx} % creation of more complex tables
\usepackage{longtable} % tables can span multiple pages
\usepackage{array} % allow elements of tabular environment to have vertical alignment, e.g., center alignment.
\usepackage{nameref} % make it possible to reference by name
\usepackage{hyperref} % allow hiperlinks (links to other document parts and extern links)
\usepackage{etoc} % used for generation of section local table of contents
\usepackage{placeins} % defines the \FloatBarrier command
\usepackage{xcolor} % for modifying box color
\usepackage{adjustbox} % for modifying box parameters
\usepackage{textcomp} % for having the degree symbol
\usepackage{listings} % for code snippets

% configure a listing environment for code snippets
\lstnewenvironment{code}
    {\lstset{linewidth=\linewidth,
             backgroundcolor=\color{white},
             frame=single,basicstyle=\normalsize\ttfamily\color{black},
             breaklines=true
             }}
    {}

% Define graphics path
\graphicspath{{figs/}}

% Configure the cross reference hyper links color
\hypersetup{
    colorlinks=true,
    linkcolor=blue,
}

\newcolumntype{C}{>{\centering\arraybackslash}X} % new column type for tabularx
                         % centered (\centering), adjust width in order to fill table width (X type)

% Configure header in 'titlepage'
\pagestyle{fancy}
\lhead{\includegraphics[width=4.5cm]{logo_cnpem}}
\rhead{\includegraphics[width=4cm]{logo_lnls}}
\renewcommand{\headrulewidth}{0pt}
\setlength{\headheight}{52pt}
% Clean footer
\fancyfoot{}

% increase table height factor a little bit (taller cells)
\renewcommand{\arraystretch}{1.5}

%==== Begin DOCUMENT ====
\begin{document}

%--- Begin title page ---
\begin{titlepage}

% Add header to this page
\thispagestyle{fancy}

% Center elements
\begin{center}

% title of title page
\topskip0pt % perfectly centered
\vspace*{\fill}
\textbf{\Huge Galil DMC 30017 EPICS IOC User Guide}\\[20pt]
\textbf{\Huge Version 1.0}\\[20pt]
\textbf{\Huge May/2020}
\vspace*{\fill}

% footer of title page
\vfill
\textbf{Beam Diagnostics Group (DIGS)}\\[5pt]
\textbf{Brazilian Synchrotron Light Laboratory (LNLS)}\\[5pt]
\textbf{Brazilian Center for Research in Energy and Materials (CNPEM)}
\end{center}

\end{titlepage}
%--- End of title page ---

\newpage
\pagestyle{plain} % restore default page style

%--- About this manual ---
\paragraph{}{\Large\bfseries About this manual}

\paragraph{} This manual provides an overview of the Galil DMC 30017 EPICS IOC support. It is assumed that the reader is familiar with the basics of EPICS.

%--- Table of contents ---
\tableofcontents

\newpage
%--- Section: Introduction ---
\section{Introduction}

\paragraph{} The aim of this IOC is to provide a set of already configured files for using the \href{https://github.com/motorapp/Galil-3-0}{Galil-3-0 EPICS driver} to control a DMC 30017. This IOC includes basic configurations regarding the controller type and aliases for the driver PVs that follow Sirius naming convention.

\paragraph{} The Galil DMC 30017 IOC repository is \url{https://github.com/lnls-dig/galil-dmc30017-epics-ioc}.

%--- Section: Installation ---
\section{Installation}

    \paragraph{Shortcut} It is possible to run a \emph{docker image} containing the IOC and all of its dependencies. Go to the \hyperref[sec:run-with-docker]{Running the IOC from a docker container} section to see how to do it.

    \subsection{Installing without docker}

        \paragraph{Before everything} Make sure you have EPICS Base 3.15 and the following EPICS modules installed (older versions might work but are not recommended):

        \begin{itemize}
          \item autosave R5-9
          \item seq 2-2-6
          \item sscan R2-11-1
          \item calc R3-7
          \item asyn R4-33
          \item busy R1-7
          \item motor R6-10
          \item ipac 2-15
          \item Galil-3-0 V3-6 (driver)
        \end{itemize} 

        \paragraph{} Clone the IOC repository and checkout the desired commit (in case you are not using master).

            \vspace{1mm}
            \begin{code}
git clone https://github.com/lnls-dig/galil-dmc30017-epics-ioc.git <install location>
cd <install location>
git checkout <commit>
            \end{code}
            \vspace{1mm}

        \paragraph{} Create a \emph{RELEASE.local} file inside the \emph{configure/} directory and add the paths to EPICS Base and external modules. An example file is available at \emph{configure/RELEASE.local.example}.

            \vspace{1mm}
            \begin{code}
echo 'EPICS_BASE=<path to EPICS Base>' > configure/RELEASE.local
echo 'AUTOSAVE=<path to autosave>' >> configure/RELEASE.local
echo 'SNCSEQ=<path to seq>' >> configure/RELEASE.local
echo 'SSCAN=<path to sscan>' >> configure/RELEASE.local
echo 'CALC=<path to calc>' >> configure/RELEASE.local
echo 'ASYN=<path to asyn>' >> configure/RELEASE.local
echo 'BUSY=<path to busy>' >> configure/RELEASE.local
echo 'MOTOR=<path to motor>' >> configure/RELEASE.local
echo 'IPAC=<path to ipac>' >> configure/RELEASE.local
echo 'GALIL=<path to Galil-3-0 driver>' >> configure/RELEASE.local
            \end{code}
            \vspace{1mm}

        \paragraph{} From the IOC $<$TOP$>$ directory, run make:

            \vspace{1mm}
            \begin{code}
make
            \end{code}
            \vspace{1mm}

%--- Section: Running the IOC without docker ---
\section{Running the IOC without docker}

    After successfully building the IOC, from the IOC top directory, run:

        \vspace{1mm}
        \begin{code}
cd iocBoot/iocGalilDmc30017 &&
./runGalilDmc30017.sh <options>
        \end{code}
        \vspace{1mm}

    The available options can be listed by passing any invalid option to the startup script, such as:

        \vspace{1mm}
        \begin{code}
./runDiffCtrl.sh -h
        \end{code}
        \vspace{1mm}

    or can be consulted in the README.md file at the IOC $<$TOP$>$ directory or project page \url{https://github.com/lnls-dig/galil-dmc30017-epics-ioc}.

%--- Section: Running the IOC from a docker container ---
\section{Running the IOC from a docker container}\label{sec:run-with-docker}

    \paragraph{} Make sure you have docker installed and that you understand the basics of creating, running and acessing a container, and downloading an image.

    \subsection{Downloading the image (IOC + dependencies)}

        \paragraph{} The image named \textbf{lnlsdig/galil-dmc30017-epics-ioc} contains the compiled IOC and dependencies, and can be downloaded to your machine by running:

        \vspace{1mm}
        \begin{code}
docker pull lnlsdig/galil-dmc30017-epics-ioc:<IOC_TAG>
        \end{code}
        \vspace{1mm}

        where \emph{$<$IOC\_TAG$>$} should be replaced by the desired IOC version.

        \paragraph{} The IOC version list can be found at \url{https://hub.docker.com/r/lnlsdig/galil-dmc30017-epics-ioc/tags}.

        After the IOC image is downloaded, you can instante it by running a container that uses it.

    \subsection{Running a docker container with that image}

        \paragraph{} One way to run the container is to do:

        \vspace{1mm}
        \begin{code}
docker run -it --network host --restart always --name <CONTAINER_NAME> lnlsdig/galil-dmc30017-epics-ioc:<IOC_TAG> -P <PREFIX1> -R <PREFIX2> <other options>
        \end{code}
        \vspace{1mm}

        Which is going to run the container starting at its default entrypoint, that is, the IOC startup script. The parameter $<$IOC\_TAG$>$ should be replaced by the IOC version. All the arguments after it are passed to the IOC startup script when run in this way. The $<$PREFIX1$>$ and $<$PREFIX2$>$ are the IOC EPICS PVs prefix, combined as follows: \emph{PV\_PREFIX = PREFIX1 + PREFIX2}. The other startup script arguments should be provided as well. They are listed in the IOC README.md file.

        \paragraph{} \textbf{If you run this command before downloading the image, the image is automatically downloaded, provided there is internet connection}.

        \paragraph{} Below you find an explanation for each flag we used in the above command.

        \begin{itemize}
          \item -it: interactive (STDIN open) and allocate pseudo-TTY. Since we did not change the container entrypoint, your terminal is going to connect to the IOC shell.
          \item --network host: configures the container to use the host's network stack inside the container. It is the simplest configuration to use, although it provides no isolation.
          \item --restart always: restarts the container automatically when it exits.
          \item --name $<$CONTAINER\_NAME$>$: configures the container name as $<$CONTAINER\_NAME$>$.
        \end{itemize}

        \paragraph{} In order to stop the docker container, i.e., the IOC, run:

        \vspace{1mm}
        \begin{code}
docker stop <CONTAINER_NAME>
        \end{code}
        \vspace{1mm}

        \paragraph{} If you want to delete the container, after stopping it, run:

        \vspace{1mm}
        \begin{code}
docker rm <CONTAINER_NAME>
        \end{code}
        \vspace{1mm}

        \paragraph{} You can also run the contaier with the \emph{--rm} flag to clean up the container and remove the file system when it exits.

%--- Section: Alises for Driver PVs ---
\section{Alises for Driver PVs}\label{sec:dev-reference-frame}

    This document is still under development.

%--- Section: Auto Settings ---
\section{Auto Settings}

    This document is still under development.

%--- Section: Startup Configuration ---
\section{Startup Configuration}

    This document is still under development.

%--- Section: CS-Studio OPI Overview ---
\section{CS-Studio OPI Overview}

    This document is still under development.

\newpage
\section{Process Variables Description}\label{sec:process-variables}

    Each IOC instance should add a prefix to the process variables indicating which device it controls.

    % Process Variables description table
    \begin{longtable}{| m{4.5cm} m{2.5cm}  m{7.0cm} |}
        \caption{Application Process Variables} \\ \hline
        \bfseries Name & \bfseries Data Type & \bfseries Description \label{tab:PV-description} \endfirsthead
        \caption{Application Process Variables} \\ \hline
        \bfseries Name & \bfseries Data Type & \bfseries Description \endhead \hline
        % --- row ---
        PV\_NAME & DATA\_TYPE & \begin{tabular}{@{}m{6cm}@{}}
                            DESCRIPTION
            \end{tabular} \hypertarget{}{}\\ \hline
        % --- row ---
         &  & \begin{tabular}{@{}m{6cm}@{}}
                .
            \end{tabular} \hypertarget{}{}\\ \hline
        % --- row ---
         &  & \begin{tabular}{@{}m{6cm}@{}}
                .
            \end{tabular} \hypertarget{}{}\\ \hline
    \end{longtable}

\end{document}
\grid


\documentclass[openany]{article}
\usepackage[a4paper,margin=1in,bottom=1.5in]{geometry} % define margins. Bottom margin is used to lift a little bit the page number.
\usepackage[english]{babel} % document language is english
\usepackage{graphicx} % for including images
\usepackage[export]{adjustbox}
\usepackage{fancyhdr} % used for creating headers and footers. only used in title page in this document.
\usepackage{tabularx} % creation of more complex tables
\usepackage{longtable} % tables can span multiple pages
\usepackage{array} % allow elements of tabular environment to have vertical alignment, e.g., center alignment.
\usepackage{nameref} % make it possible to reference by name
\usepackage{hyperref} % allow hiperlinks (links to other document parts and extern links)
\usepackage{etoc} % used for generation of section local table of contents
\usepackage{placeins} % defines the \FloatBarrier command
\usepackage[dvipsnames]{xcolor} % for modifying box color
\usepackage{tikz} % for drawing (currently not used).
\usepackage{adjustbox} % for modifying box parameters
\usepackage{textcomp} % for having the degree symbol
\usepackage{listings} % for code snippets

% configure a listing environment for code snippets
\lstnewenvironment{code}
    {\lstset{linewidth=\linewidth,
             backgroundcolor=\color{white},
             frame=single,basicstyle=\normalsize\ttfamily\color{black},
             breaklines=true
             }}
    {}

% Define graphics path
\graphicspath{{figs/}}

% Configure the cross reference hyper links color
\hypersetup{
    colorlinks=true,
    linkcolor=blue,
}

\newcolumntype{C}{>{\centering\arraybackslash}X} % new column type for tabularx
                         % centered (\centering), adjust width in order to fill table width (X type)

% Configure header in 'titlepage'
\pagestyle{fancy}
\lhead{\includegraphics[width=4.5cm]{logo_cnpem}}
\rhead{\includegraphics[width=4cm]{logo_lnls}}
\renewcommand{\headrulewidth}{0pt}
\setlength{\headheight}{52pt}
% Clean footer
\fancyfoot{}

% increase table height factor a little bit (taller cells)
\renewcommand{\arraystretch}{1.5}

%==== Begin DOCUMENT ====
\begin{document}

%--- Begin title page ---
\begin{titlepage}

% Add header to this page
\thispagestyle{fancy}

% Center elements
\begin{center}

% title of title page
\topskip0pt % perfectly centered
\vspace*{\fill}
\textbf{\Huge Galil DMC 30017 EPICS IOC User Guide}\\[20pt]
\textbf{\Huge Version 1.0}\\[20pt]
\textbf{\Huge May/2020}
\vspace*{\fill}

% footer of title page
\vfill
\textbf{Beam Diagnostics Group (DIGS)}\\[5pt]
\textbf{Brazilian Synchrotron Light Laboratory (LNLS)}\\[5pt]
\textbf{Brazilian Center for Research in Energy and Materials (CNPEM)}
\end{center}

\end{titlepage}
%--- End of title page ---

\newpage
\pagestyle{plain} % restore default page style

%--- About this manual ---
\paragraph{}{\Large\bfseries About this manual}

\paragraph{} This manual provides an overview of the Galil DMC 30017 EPICS IOC support used at Sirius. It is assumed that the reader is familiar with the basics of EPICS.

%--- Table of contents ---
\tableofcontents

\newpage
%--- Section: Introduction ---
\section{Introduction}

\paragraph{} The aim of this IOC is to provide a set of already configured files for using the \href{https://github.com/motorapp/Galil-3-0}{Galil-3-0 EPICS driver} to control a DMC 30017, in the context of the applications at Sirius. This IOC includes basic configurations regarding the controller type and aliases following the Sirius naming convention for the driver main PVs.

\paragraph{} The Galil DMC 30017 IOC repository is \url{https://github.com/lnls-dig/galil-dmc30017-epics-ioc}.

%--- Section: Installation ---
\section{Installation}

    \paragraph{Shortcut} It is possible to run a \emph{docker image} containing the IOC and all of its dependencies. Go to the \hyperref[sec:run-with-docker]{Running the IOC from a docker container} section to see how to do it.

    \subsection{Installing without docker}

        \paragraph{Before everything} Make sure you have EPICS Base 3.15 and the following EPICS modules installed (older versions might work but are not recommended):

        \begin{itemize}
          \item autosave R5-9
          \item seq 2-2-6
          \item sscan R2-11-1
          \item calc R3-7
          \item asyn R4-33
          \item busy R1-7
          \item motor R6-10
          \item ipac 2-15
          \item Galil-3-0 V3-6 (driver)
        \end{itemize} 

        \paragraph{} Clone the IOC repository and checkout the desired commit (in case you are not using master).

            \vspace{1mm}
            \begin{code}
git clone https://github.com/lnls-dig/galil-dmc30017-epics-ioc.git <install location>
cd <install location>
git checkout <commit>
            \end{code}
            \vspace{1mm}

        \paragraph{} Create a \emph{RELEASE.local} file inside the \emph{configure/} directory and add the paths to EPICS Base and external modules. An example file is available at \emph{configure/RELEASE.local.example}.

            \vspace{1mm}
            \begin{code}
echo 'EPICS_BASE=<path to EPICS Base>' > configure/RELEASE.local
echo 'AUTOSAVE=<path to autosave>' >> configure/RELEASE.local
echo 'SNCSEQ=<path to seq>' >> configure/RELEASE.local
echo 'SSCAN=<path to sscan>' >> configure/RELEASE.local
echo 'CALC=<path to calc>' >> configure/RELEASE.local
echo 'ASYN=<path to asyn>' >> configure/RELEASE.local
echo 'BUSY=<path to busy>' >> configure/RELEASE.local
echo 'MOTOR=<path to motor>' >> configure/RELEASE.local
echo 'IPAC=<path to ipac>' >> configure/RELEASE.local
echo 'GALIL=<path to Galil-3-0 driver>' >> configure/RELEASE.local
            \end{code}
            \vspace{1mm}

        \paragraph{} From the IOC $<$TOP$>$ directory, run make:

            \vspace{1mm}
            \begin{code}
make
            \end{code}
            \vspace{1mm}

%--- Section: Running the IOC without docker ---
\section{Running the IOC without docker}

    After successfully building the IOC, from the IOC top directory, run:

        \vspace{1mm}
        \begin{code}
cd iocBoot/iocGalilDmc30017 &&
./runGalilDmc30017.sh <options>
        \end{code}
        \vspace{1mm}

    The available options can be listed by passing any invalid option to the startup script, such as:

        \vspace{1mm}
        \begin{code}
./runDiffCtrl.sh -h
        \end{code}
        \vspace{1mm}

    or can be consulted in the README.md file at the IOC $<$TOP$>$ directory or project page \url{https://github.com/lnls-dig/galil-dmc30017-epics-ioc}.

%--- Section: Running the IOC from a docker container ---
\section{Running the IOC from a docker container}\label{sec:run-with-docker}

    \paragraph{} Make sure you have docker installed and that you understand the basics of creating, running and acessing a container, and downloading an image.

    \subsection{Downloading the image (IOC + dependencies)}

        \paragraph{} The image named \textbf{lnlsdig/galil-dmc30017-epics-ioc} contains the compiled IOC and dependencies, and can be downloaded to your machine by running:

        \vspace{1mm}
        \begin{code}
docker pull lnlsdig/galil-dmc30017-epics-ioc:<IOC_TAG>
        \end{code}
        \vspace{1mm}

        where \emph{$<$IOC\_TAG$>$} should be replaced by the desired IOC version.

        \paragraph{} The IOC version list can be found at \url{https://hub.docker.com/r/lnlsdig/galil-dmc30017-epics-ioc/tags}.

        After the IOC image is downloaded, you can instante it by running a container that uses it.

    \subsection{Running a docker container with that image}

        \paragraph{} One way to run the container is to do:

        \vspace{1mm}
        \begin{code}
docker run -it --network host --restart always --name <CONTAINER_NAME> lnlsdig/galil-dmc30017-epics-ioc:<IOC_TAG> -P <PREFIX1> -R <PREFIX2> <other options>
        \end{code}
        \vspace{1mm}

        Which is going to run the container starting at its default entrypoint, that is, the IOC startup script. The parameter $<$IOC\_TAG$>$ should be replaced by the IOC version. All the arguments after it are passed to the IOC startup script when run in this way. The $<$PREFIX1$>$ and $<$PREFIX2$>$ are the IOC EPICS PVs prefix, combined as follows: \emph{PV\_PREFIX = PREFIX1 + PREFIX2}. The other startup script arguments should be provided as well. They are listed in the IOC README.md file.

        \paragraph{} \textbf{If you run this command before downloading the image, the image is automatically downloaded, provided there is internet connection}.

        \paragraph{} Below you find an explanation for each flag we used in the above command.

        \begin{itemize}
          \item -it: interactive (STDIN open) and allocate pseudo-TTY. Since we did not change the container entrypoint, your terminal is going to connect to the IOC shell.
          \item --network host: configures the container to use the host's network stack inside the container. It is the simplest configuration to use, although it provides no isolation.
          \item --restart always: restarts the container automatically when it exits.
          \item --name $<$CONTAINER\_NAME$>$: configures the container name as $<$CONTAINER\_NAME$>$.
        \end{itemize}

        \paragraph{} In order to stop the docker container, i.e., the IOC, run:

        \vspace{1mm}
        \begin{code}
docker stop <CONTAINER_NAME>
        \end{code}
        \vspace{1mm}

        \paragraph{} If you want to delete the container, after stopping it, run:

        \vspace{1mm}
        \begin{code}
docker rm <CONTAINER_NAME>
        \end{code}
        \vspace{1mm}

        \paragraph{} You can also run the contaier with the \emph{--rm} flag to clean up the container and remove the file system when it exits.

%--- Section: Alises for Driver PVs ---
\section{Alises for Driver PVs}\label{sec:dev-reference-frame}

    This document is still under development.

%--- Section: Auto Settings ---
\section{Auto Settings}

    This document is still under development.

%--- Section: Startup Configuration ---
\section{Startup Configuration}

    This document is still under development.

%--- Section: CS-Studio OPI Overview ---
\section{CS-Studio OPI Overview}

    This document is still under development.

\newpage
\section{Process Variables Description}\label{sec:process-variables}

    Each IOC instance should add a prefix to the process variables indicating which device it controls.

    % Process Variables description table
    \begin{longtable}{| m{4.5cm} m{2.5cm}  m{8.5cm} |}
        \caption{Application Process Variables} \\ \hline
        \bfseries Name & \bfseries Data Type & \bfseries Description \label{tab:PV-description} \endfirsthead
        \caption{Application Process Variables} \\ \hline
        \bfseries Name & \bfseries Data Type & \bfseries Description \endhead \hline
        % --- row ---
        PV\_NAME & DATA\_TYPE & \begin{tabular}{@{}m{6cm}@{}}
                            DESCRIPTION
            \end{tabular} \hypertarget{pv:mtr}{}\\ \hline
        % --- row ---
        Mtr & Motor & \begin{tabular}{@{}m{6cm}@{}}
                \textbf{\color{blue} Alias} for \$(PV\_PREFIX)\$(M), the motor record associated to the \$(M) axis (there is only 1 axis available in the DMC 30017). Most aspects of motor control as speed, position and direction can be controlled by the fields of this PV. More information about it can be found in the EPICS Motor Record documentation. In this IOC the macro M="A", by default. As a quick reference, the main motor record fields are listed below.
            \end{tabular} \hypertarget{pv:mtr-mres}{}\\ \hline
        % --- row ---
        Mtr.MRES & Float & \begin{tabular}{@{}m{6cm}@{}}
                \textbf{\color{ForestGreen} Motor record field} Motor resolution, i.e., distance per step, in engineering units. If the driver has microstepping enabled (which is always true for the DMC 30017), then the microstep resolution should be used. When this field is set, motion, in engineering units, per revolution is updated to match the current number of steps per revolution, i.e., UREV = MRES*SREV. More information about it can be found in the EPICS Motor Record documentation.
            \end{tabular} \hypertarget{pv:mtr-srev}{}\\ \hline
        % --- row ---
        Mtr.SREV & Long & \begin{tabular}{@{}m{6cm}@{}}
                \textbf{\color{ForestGreen} Motor record field} Number of motor steps per revolution. If the driver has microstepping enabled (which is always true for the DMC 30017), then the microstep count per revolution should be used. When this field is set, motor step resolution is automatically updated to match the current motion, in engineering units, per revolution, i.e., MRES = UREV/SREV. More information about it can be found in the EPICS Motor Record documentation.
            \end{tabular} \hypertarget{pv:mtr-urev}{}\\ \hline
        % --- row ---
        Mtr.UREV & Float & \begin{tabular}{@{}m{6cm}@{}}
                \textbf{\color{ForestGreen} Motor record field} Motion, in engineering units, per motor revolution. When this field is set, motor step resolution is automatically updated to match the current number of steps per revolution, i.e., MRES = UREV/SREV. More information about it can be found in the EPICS Motor Record documentation.
            \end{tabular} \hypertarget{pv:mtr-velo}{}\\ \hline
        % --- row ---
        Mtr.VELO & Float & \begin{tabular}{@{}m{6cm}@{}}
                \textbf{\color{ForestGreen} Motor record field} Velocity the motor should have after finishing acceleration, in engineering units. More information about it can be found in the EPICS Motor Record documentation.
            \end{tabular} \hypertarget{pv:mtr-vmax}{}\\ \hline
        % --- row ---
        Mtr.VMAX & Float & \begin{tabular}{@{}m{6cm}@{}}
                \textbf{\color{ForestGreen} Motor record field} User configurable maximum velocity the VELO field can be configured to, in engineering units. More information about it can be found in the EPICS Motor Record documentation.
            \end{tabular} \hypertarget{pv:mtr-accl}{}\\ \hline
        % --- row ---
        Mtr.ACCL & Float & \begin{tabular}{@{}m{6cm}@{}}
                \textbf{\color{ForestGreen} Motor record field} Acceleration time, in seconds. This is the configured time period until the motor reaches the specified velocity, starting from rest. More information about it can be found in the EPICS Motor Record documentation.
            \end{tabular} \hypertarget{pv:mtr-set}{}\\ \hline
        % --- row ---
        Mtr.SET & Enum: Use (0), Set (1) & \begin{tabular}{@{}m{6cm}@{}}
                \textbf{\color{ForestGreen} Motor record field} Toggle between move and calibration. When \emph{Use} is selected, writes to the motor VAL field cause it to move. When \emph{Set} is selected, writes to the motor VAL field also change the hardware position register, but do not move the motor. If the FOFF field is equal to 0, then the offset (OFF field) between hardware and user position is updated when a new position is set to VAL. More information about it can be found in the EPICS Motor Record documentation.
            \end{tabular} \hypertarget{pv:mtr-foff}{}\\ \hline
        % --- row ---
        Mtr.FOFF & Enum: Variable (0), Frozen (1) & \begin{tabular}{@{}m{6cm}@{}}
                \textbf{\color{ForestGreen} Motor record field} Freeze the "user" and "dial" coordinates difference, i.e., the offset (OFF field) is NOT automatically adjusted when a new position is set to VAL and the SET field is equal to \emph{Set}. More information about it can be found in the EPICS Motor Record documentation.
            \end{tabular} \hypertarget{pv:mtr-dir}{}\\ \hline
        % --- row ---
        Mtr.DIR & Enum: Pos (0), Neg (1) & \begin{tabular}{@{}m{6cm}@{}}
                \textbf{\color{ForestGreen} Motor record field} User direction. This field defines whether or not the user position value (set point and encoder reading) have the opposite sign with respect to the hardware values. More information about it can be found in the EPICS Motor Record documentation.
            \end{tabular} \hypertarget{pv:mtr-eres}{}\\ \hline
        % --- row ---
        Mtr.ERES & Float & \begin{tabular}{@{}m{6cm}@{}}
                \textbf{\color{ForestGreen} Motor record field} Encoder step size, in engineering units. More information about it can be found in the EPICS Motor Record documentation.
            \end{tabular} \hypertarget{pv:mtr-egu}{}\\ \hline
        % --- row ---
        Mtr.EGU & String (char[40]) & \begin{tabular}{@{}m{6cm}@{}}
                \textbf{\color{ForestGreen} Motor record field} The string corresponding to the engineering units being used (configured by the user). More information about it can be found in the EPICS Motor Record documentation.
            \end{tabular} \hypertarget{pv:mtr-hlm}{}\\ \hline
        % --- row ---
        Mtr.HLM & Float & \begin{tabular}{@{}m{6cm}@{}}
                \textbf{\color{ForestGreen} Motor record field} High position limit. This is a software high limit to the allowed motor command positions. More information about it can be found in the EPICS Motor Record documentation.
            \end{tabular} \hypertarget{pv:mtr-llm}{}\\ \hline
        % --- row ---
        Mtr.LLM & Float & \begin{tabular}{@{}m{6cm}@{}}
                \textbf{\color{ForestGreen} Motor record field} Low position limit. This is a software low limit to the allowed motor command positions. More information about it can be found in the EPICS Motor Record documentation.
            \end{tabular} \hypertarget{pv:mtr-homf}{}\\ \hline
        % --- row ---
        Mtr.HOMF & Short & \begin{tabular}{@{}m{6cm}@{}}
                \textbf{\color{ForestGreen} Motor record field} When set to 1, home the motor forward. The user direction (DIR field) has no effect in the meaning of "forward" in homing. More information about it can be found in the EPICS Motor Record documentation.
            \end{tabular} \hypertarget{pv:mtr-homr}{}\\ \hline
        % --- row ---
        Mtr.HOMR & Short & \begin{tabular}{@{}m{6cm}@{}}
                \textbf{\color{ForestGreen} Motor record field} When set to 1, home the motor backwards. Theuser direction (DIR field) has no effect in the meaning of "backwards" in homing. More information about it can be found in the EPICS Motor Record documentation.
            \end{tabular} \hypertarget{pv:mtr-hvel}{}\\ \hline
        % --- row ---
        Mtr.HVEL & Float & \begin{tabular}{@{}m{6cm}@{}}
                \textbf{\color{ForestGreen} Motor record field} Homing velocity. The velocity used during a homing procedure. This value cannot be set to a value greater than the value of the VMAX field. More information about it can be found in the EPICS Motor Record documentation.
            \end{tabular} \hypertarget{pv:mtr-athm}{}\\ \hline
        % --- row ---
        Mtr.ATHM & Short & \begin{tabular}{@{}m{6cm}@{}}
                \textbf{\color{ForestGreen} Motor record field} The home switch state. It indicates 1 while the home switch is being pressed and 0 when it is not active. More information about it can be found in the EPICS Motor Record documentation.
            \end{tabular} \hypertarget{pv:mtr-bdst}{}\\ \hline
        % --- row ---
        Mtr.BDST & Float & \begin{tabular}{@{}m{6cm}@{}}
                \textbf{\color{ForestGreen} Motor record field} Backlash distance, in engineering units. When the motor is commanded to move a distance greater tha  the magnitude of the \emph{backlash distace} or to move in a direction opposite to the sign of the \emph{backlash distance}, the motor will first move to $target - backlash$ at velocity \emph{Mtr.VELO} and acceleration \emph{Mtr.ACCL}, then move to the target position at the \emph{backlash speed} and \emph{backlash acceleration}. More information about it can be found in the EPICS Motor Record documentation.
            \end{tabular} \hypertarget{pv:mtr-bvel}{}\\ \hline
        % --- row ---
        Mtr.BVEL & Float & \begin{tabular}{@{}m{6cm}@{}}
                \textbf{\color{ForestGreen} Motor record field} Backlash velocity, in engineering units. This is the speed of the motor during the backlash-takeout phase, after acceleration has finished. More information about it can be found in the EPICS Motor Record documentation.
            \end{tabular} \hypertarget{pv:mtr-bacc}{}\\ \hline
        % --- row ---
        Mtr.BACC & Float & \begin{tabular}{@{}m{6cm}@{}}
                \textbf{\color{ForestGreen} Motor record field} The time period, in seconds, of the acceleration (or deceleration) phase of a backlash-takeout move. More information about it can be found in the EPICS Motor Record documentation.
            \end{tabular} \hypertarget{pv:mtr-off}{}\\ \hline
        % --- row ---
        Mtr.OFF & Float & \begin{tabular}{@{}m{6cm}@{}}
                \textbf{\color{ForestGreen} Motor record field} User offset. This PV defines the offset between hardware (stored set point and encoder reading) and user position: $userPos = Mtr.DIR * HwdPos + Offset$. More information about it can be found in the EPICS Motor Record documentation.
            \end{tabular} \hypertarget{pv:mtr-ntm}{}\\ \hline
        % --- row ---
        Mtr.NTM & Enum: NO (0), YES (1) & \begin{tabular}{@{}m{6cm}@{}}
                \textbf{\color{ForestGreen} Motor record field} New target monitor. This PV defines the motor behavior when a position command is received during an ongoing movement. If this PV is set to \emph{YES}, then the motor will stop the current move and immediately go to the next commanded position. More information about it can be found in the EPICS Motor Record documentation.
            \end{tabular} \hypertarget{pv:mtr-ueip}{}\\ \hline
        % --- row ---
        Mtr.UEIP & Enum: No (0), Yes (1) & \begin{tabular}{@{}m{6cm}@{}}
                \textbf{\color{ForestGreen} Motor record field} \emph{Use Encoder If Present}. This PV defines if the encoder signal should be used, as long as it is available, instead of the hardware step count. More information about it can be found in the EPICS Motor Record documentation.
            \end{tabular} \hypertarget{pv:mtr-rtry}{}\\ \hline
        % --- row ---
        Mtr.RTRY & Short & \begin{tabular}{@{}m{6cm}@{}}
                \textbf{\color{ForestGreen} Motor record field} Maximum Retry Count. This is the maximum number of times the motor will try to reach the commanded position. If the motor fails to reach the commanded position after all available retries, then it finishes motion and the \emph{Mtr.MISS} PV is set to 1. More information about it can be found in the EPICS Motor Record documentation.
            \end{tabular} \hypertarget{pv:mtr-rdbd}{}\\ \hline
        % --- row ---
        Mtr.RDBD & Float & \begin{tabular}{@{}m{6cm}@{}}
                \textbf{\color{ForestGreen} Motor record field} Retry Deadband. After a motion is complete, if the difference between the current and desired positions is greater than the \emph{retry deadband}, then the motor will try again to reach the target position, as long as the \emph{maximum retry count} has not been reached. More information about it can be found in the EPICS Motor Record documentation.
            \end{tabular} \hypertarget{pv:mtr-rmod}{}\\ \hline
        % --- row ---
        Mtr.RMOD & Enum: Unity (0), Arthmetic (1), Geometric (2), In-Position (3) & \begin{tabular}{@{}m{6cm}@{}}
                \textbf{\color{ForestGreen} Motor record field} Retry Mode. This PV selects the retry move distance calculation method.
                \begin{itemize}
                    \item 0: Unity - motor moves by the difference between desired and current positions.
                    \item 1: Arthmetic - motor retry moves follow an arithmetic sequence: $positionDiff \times 1$, $positionDiff \times\frac{maxRetryCount - 1}{maxRetryCount}$, ... , $positionDiff \times\frac{1}{maxRetryCount}$.
                    \item 2: Geometric - motor retry moves follow a geometric sequence: $positionDiff \times 1$, $positionDiff \times\frac{1}{2}$, ... , $positionDiff \times\frac{1}{2^{n}}$, where n = \emph{max retry count}.
                    \item 3: In Position - This mode should be used only with servo motors. This mode does not send new motion commands, it only waits the motor to reach an acceptable position by letting a previous move continue until the difference in position becomes small enough.
                \end{itemize}
More information about it can be found in the EPICS Motor Record documentation.
            \end{tabular} \hypertarget{pv:mtr-miss}{}\\ \hline
        % --- row ---
        Mtr.MISS & Short & \begin{tabular}{@{}m{6cm}@{}}
                \textbf{\color{ForestGreen} Motor record field} \emph{Motor is out of retries}. When the motor fails to reach the commanded position after the maximum allowed number of retries, this PV is set to 1. More information about it can be found in the EPICS Motor Record documentation.
            \end{tabular} \hypertarget{pv:controller-model-mon}{}\\ \hline
        % --- row ---
        ControllerModel-Mon & String (char[40]) & \begin{tabular}{@{}m{6cm}@{}}
                \textbf{\color{blue} Alias} for \$(PV\_PREFIX)MODEL\_MON. The controller model information.
            \end{tabular} \hypertarget{pv:controller-addr-mon}{}\\ \hline
        % --- row ---
        ControllerAddr-Mon & String (char[40]) & \begin{tabular}{@{}m{6cm}@{}}
                \textbf{\color{blue} Alias} for \$(PV\_PREFIX)ADDRESS\_MON. The controller IP address or DNS name.
            \end{tabular} \hypertarget{pv:lim-sw-type}{}\\ \hline
        % --- row ---
        LimSwType-Sel & Enum: NO (0), NC (1) & \begin{tabular}{@{}m{6cm}@{}}
                \textbf{\color{blue} Alias} for \$(PV\_PREFIX)LIMITTYPE\_CMD. The desired limit switch configuration type: \emph{normally open} or \emph{normally closed}.
            \end{tabular} \hypertarget{}{}\\ \hline
        % --- row ---
        LimSwType-Sts & Enum: NO (0), NC (1) & \begin{tabular}{@{}m{6cm}@{}}
                \textbf{\color{blue} Alias} for \$(PV\_PREFIX)LIMITTYPE\_STATUS. The status of the limit switch type configuration.
            \end{tabular} \hypertarget{pv:home-sw-type}{}\\ \hline
        % --- row ---
        HomeSwType-Sel & Enum: NO (0), NC (1) & \begin{tabular}{@{}m{6cm}@{}}
                \textbf{\color{blue} Alias} for \$(PV\_PREFIX)HOMETYPE\_CMD. The desired home switch configuration type: \emph{normally open} or \emph{normally closed}.
            \end{tabular} \hypertarget{}{}\\ \hline
        % --- row ---
        HomeSwType-Sts & Enum: NO (0), NC (1) & \begin{tabular}{@{}m{6cm}@{}}
                \textbf{\color{blue} Alias} for \$(PV\_PREFIX)HOMETYPE\_STATUS. The status of the home switch type configuration.
            \end{tabular} \hypertarget{pv:mac-addr-mon}{}\\ \hline
        % --- row ---
        MacAddr-Mon & String (char[40]) & \begin{tabular}{@{}m{6cm}@{}}
                \textbf{\color{blue} Alias} for \$(PV\_PREFIX)ETHADDR\_MON. The controller MAC address.
            \end{tabular} \hypertarget{pv:serial-nr-mon}{}\\ \hline
        % --- row ---
        SerialNr-Mon & String (char[40]) & \begin{tabular}{@{}m{6cm}@{}}
                \textbf{\color{blue} Alias} for \$(PV\_PREFIX)SERIALNUM\_MON. The controller serial number.
            \end{tabular} \hypertarget{pv:comm-err-mon}{}\\ \hline
        % --- row ---
        CommErr-Mon & Enum: OK (0), Error (1) & \begin{tabular}{@{}m{6cm}@{}}
                \textbf{\color{blue} Alias} for \$(PV\_PREFIX)COMMERR\_STATUS. The status of the communication between IOC and controller.
            \end{tabular} \hypertarget{pv:controller-start-mon}{}\\ \hline
        % --- row ---
        ControllerStart-Mon & Enum: Error (0), OK (1)  & \begin{tabular}{@{}m{6cm}@{}}
                \textbf{\color{blue} Alias} for \$(PV\_PREFIX)START\_STATUS. The controller start status.
            \end{tabular} \hypertarget{pv:deferred-move}{}\\ \hline
        % --- row ---
        DeferredMove-Sel & Enum: Go (0), Defer (1) & \begin{tabular}{@{}m{6cm}@{}}
                \textbf{\color{blue} Alias} for \$(PV\_PREFIX)DEFER\_CMD. Deferred moves configuration.
            \end{tabular} \hypertarget{}{}\\ \hline
        % --- row ---
        DeferredMove-Sts & Enum: Go (0), Defer (1) & \begin{tabular}{@{}m{6cm}@{}}
                \textbf{\color{blue} Alias} for \$(PV\_PREFIX)DEFER\_STATUS. The status of the deferred move configuration.
            \end{tabular} \hypertarget{pv:upload-user-array-cmd}{}\\ \hline
        % --- row ---
        UploadUserArray-Cmd & Enum: Upload (0), Upload (1) & \begin{tabular}{@{}m{6cm}@{}}
                \textbf{\color{blue} Alias} for \$(PV\_PREFIX))UPLOAD\_CMD. Command to upload the user array. The command is sent when the PV is set to 1 or 0.
            \end{tabular} \hypertarget{pv:upload-user-array-mon}{}\\ \hline
        % --- row ---
        UploadUserArray-Mon & Enum: Idle (0), Uploading (1) & \begin{tabular}{@{}m{6cm}@{}}
                \textbf{\color{blue} Alias} for \$(PV\_PREFIX)UPLOAD\_STATUS. The status of the user array upload operation.
            \end{tabular} \hypertarget{pv:deferred-move-mode}{}\\ \hline
        % --- row ---
        DeferredMoveMode-Sel & Enum: Sync start only (0), Sync start/stop (1) & \begin{tabular}{@{}m{6cm}@{}}
                \textbf{\color{blue} Alias} for \$(PV\_PREFIX)DEFER\_MODE\_CMD. Deferred move mode selection.
            \end{tabular} \hypertarget{}{}\\ \hline
        % --- row ---
        DeferredMoveMode-Sts & Enum: Sync start only (0), Sync start/stop (1) & \begin{tabular}{@{}m{6cm}@{}}
                \textbf{\color{blue} Alias} for \$(PV\_PREFIX)DEFER\_MODE\_STATUS. Status of deferred move mode.
            \end{tabular} \hypertarget{pv:controller-err-mon}{}\\ \hline
        % --- row ---
        ControllerErr-Mon & char[256] & \begin{tabular}{@{}m{6cm}@{}}
                \textbf{\color{blue} Alias} for \$(PV\_PREFIX)ERROR\_MON. This PV contains the last error string informed by the controller.
            \end{tabular} \hypertarget{pv:send-cmd}{}\\ \hline
        % --- row ---
        Send-Cmd & String (char[40]) & \begin{tabular}{@{}m{6cm}@{}}
                \textbf{\color{blue} Alias} for \$(PV\_PREFIX)SEND\_STR\_CMD. Send the specified command string to the controller.
            \end{tabular} \hypertarget{pv:response-mon}{}\\ \hline
        % --- row ---
        Response-Mon & char[256] & \begin{tabular}{@{}m{6cm}@{}}
                \textbf{\color{blue} Alias} for \$(PV\_PREFIX)SEND\_STR\_MON. The response string provided by the controller to a custom command sent.
            \end{tabular} \hypertarget{pv:response-val-mon}{}\\ \hline
        % --- row ---
        ResponseVal-Mon & Float & \begin{tabular}{@{}m{6cm}@{}}
                \textbf{\color{blue} Alias} for \$(PV\_PREFIX)SEND\_STRVAL\_MON. The numerical response provided by the controller to a custom command sent.
            \end{tabular} \hypertarget{pv:estall}{}\\ \hline
        % --- row ---
        Estall-Sts & Enum: Working Ok (0), Stalled (1) & \begin{tabular}{@{}m{6cm}@{}}
                \textbf{\color{blue} Alias} for \$(PV\_PREFIX)\$(M)\_ESTALL\_STATUS. In this IOC the macro M="A", by default. This PV indicates if the motor has stalled based on the encoder readings.
            \end{tabular} \hypertarget{pv:estall-time}{}\\ \hline
        % --- row ---
        EstallTime-SP & Float & \begin{tabular}{@{}m{6cm}@{}}
                \textbf{\color{blue} Alias} for \$(PV\_PREFIX)\$(M)\_ESTALLTIME\_SP. In this IOC the macro M="A", by default. This PV defines the time interval for the motor to be considered stalled.
            \end{tabular} \hypertarget{}{}\\ \hline
        % --- row ---
        EstallTime-RB & Float & \begin{tabular}{@{}m{6cm}@{}}
                \textbf{\color{blue} Alias} for \$(PV\_PREFIX)\$(M)\_ESTALLTIME\_MON. In this IOC the macro M="A", by default. This is the readback value of the encoder stall time configuration.
            \end{tabular} \hypertarget{pv:step-smooth}{}\\ \hline
        % --- row ---
        StepSmooth-SP & Float & \begin{tabular}{@{}m{6cm}@{}}
                \textbf{\color{blue} Alias} for \$(PV\_PREFIX)\$(M)\_STEPSMOOTH\_SP. In this IOC the macro M="A", by default. This PV defines de smoothing factor for output steps (greater values equals greater somoothing). This feature is most useful when working with half and full steps (the DMC 30017 has a fixed microstepping of 256).
            \end{tabular} \hypertarget{}{}\\ \hline
        % --- row ---
        StepSmooth-RB & Float & \begin{tabular}{@{}m{6cm}@{}}
                \textbf{\color{blue} Alias} for \$(PV\_PREFIX)\$(M)\_STEPSMOOTH\_MON. In this IOC the macro M="A", by default. The readback value of the smoothing factor.
            \end{tabular} \hypertarget{pv:mtr-connect-mon}{}\\ \hline
        % --- row ---
        MtrConnect-Mon & Enum: Disconnected (0), Connected (1) & \begin{tabular}{@{}m{6cm}@{}}
                \textbf{\color{blue} Alias} for \$(PV\_PREFIX)\$(M)\_MCONN\_STATUS. In this IOC the macro M="A", by default. Motor connection status.
            \end{tabular} \hypertarget{pv:use-idx}{}\\ \hline
        % --- row ---
        UseIdx-Sel & Enum: No (0), Yes (1) & \begin{tabular}{@{}m{6cm}@{}}
                \textbf{\color{blue} Alias} for \$(PV\_PREFIX)\$(M)\_UINDEX\_CMD. In this IOC the macro M="A", by default. Defines if the encoder index should be used when performing homing.
            \end{tabular} \hypertarget{}{}\\ \hline
        % --- row ---
        UseIdx-Sts & Enum: No (0), Yes (1) & \begin{tabular}{@{}m{6cm}@{}}
                \textbf{\color{blue} Alias} for \$(PV\_PREFIX)\$(M)\_UINDEX\_STATUS. In this IOC the macro M="A", by default. The status indicating if the encoder index is used during the homing operation.
            \end{tabular} \hypertarget{pv:use-sw}{}\\ \hline
        % --- row ---
        UseSw-Sel & Enum: No (0), Yes (1) & \begin{tabular}{@{}m{6cm}@{}}
                \textbf{\color{blue} Alias} for \$(PV\_PREFIX)\$(M)\_USWITCH\_CMD. In this IOC the macro M="A", by default. Configures whether or not the limit switches should be used as home switch.
            \end{tabular} \hypertarget{}{}\\ \hline
        % --- row ---
        UseSw-Sts & Enum: No (0), Yes (1) & \begin{tabular}{@{}m{6cm}@{}}
                \textbf{\color{blue} Alias} for \$(PV\_PREFIX)\$(M)\_USWITCH\_STATUS. In this IOC the macro M="A", by default. The status indicating if the limit switches are being used as the home switch.
            \end{tabular} \hypertarget{pv:user-data-mon}{}\\ \hline
        % --- row ---
        UserData-Mon & Float & \begin{tabular}{@{}m{6cm}@{}}
                \textbf{\color{blue} Alias} for \$(PV\_PREFIX)\$(M)\_USERDATA\_MON. In this IOC the macro M="A", by default.
            \end{tabular} \hypertarget{pv:user-data-deadband}{}\\ \hline
        % --- row ---
        UserDataDeadBand-SP & Float & \begin{tabular}{@{}m{6cm}@{}}
                \textbf{\color{blue} Alias} for \$(PV\_PREFIX)\$(M)\_USERDATADEADB\_SP. In this IOC the macro M="A", by default.
            \end{tabular} \hypertarget{pv:jog-after-home-sel}{}\\ \hline
        % --- row ---
        JogAfterHome-Sel & Enum: No (0), Yes (1) & \begin{tabular}{@{}m{6cm}@{}}
                \textbf{\color{blue} Alias} for \$(PV\_PREFIX)\$(M)\_JAH\_CMD. In this IOC the macro M="A", by default. Configure whether the motor should jog to a given position after homing is finished or not.
            \end{tabular} \hypertarget{}{}\\ \hline
        % --- row ---
        JogAfterHome-Sts & Enum: No (0), Yes (1) & \begin{tabular}{@{}m{6cm}@{}}
                \textbf{\color{blue} Alias} for \$(PV\_PREFIX)\$(M)\_JAH\_STATUS. In this IOC the macro M="A", by default. The status indicating if the motor jogs to a given position after hoing is finished.
            \end{tabular} \hypertarget{pv:jog-after-home-sp}{}\\ \hline
        % --- row ---
        JogAfterHome-SP & Float & \begin{tabular}{@{}m{6cm}@{}}
                \textbf{\color{blue} Alias} for \$(PV\_PREFIX)\$(M)\_JAHV\_SP. In this IOC the macro M="A", by default. Configure the position to which the motor should jog after homing is finished (jog after homing must be enabled).
            \end{tabular} \hypertarget{}{}\\ \hline
        % --- row ---
        JogAfterHome-RB & Float & \begin{tabular}{@{}m{6cm}@{}}
                \textbf{\color{blue} Alias} for \$(PV\_PREFIX)\$(M)\_JAHV\_MON. In this IOC the macro M="A", by default. The readback value of the position to which the motor should jog after homing is finished.
            \end{tabular} \hypertarget{pv:egu-after-lim}{}\\ \hline
        % --- row ---
        EguAfterLim-SP & Float & \begin{tabular}{@{}m{6cm}@{}}
                \textbf{\color{blue} Alias} for \$(PV\_PREFIX)\$(M)\_EGUAFTLIMIT\_SP. In this IOC the macro M="A", by default.
            \end{tabular} \hypertarget{}{}\\ \hline
        % --- row ---
        EguAfterLim-RB & Float & \begin{tabular}{@{}m{6cm}@{}}
                \textbf{\color{blue} Alias} for \$(PV\_PREFIX)\$(M)\_EGUAFTLIMIT\_MON. In this IOC the macro M="A", by default.
            \end{tabular} \hypertarget{pv:poll-period}{}\\ \hline
        % --- row ---
        PollPeriod-SP & Float & \begin{tabular}{@{}m{6cm}@{}}
                \textbf{\color{blue} Alias} for \$(PV\_PREFIX)\$(M)\_STATUS\_POLL\_DELAY\_CMD. In this IOC the macro M="A", by default.
            \end{tabular} \hypertarget{}{}\\ \hline
        % --- row ---
        PollPeriod-RB & Float & \begin{tabular}{@{}m{6cm}@{}}
                \textbf{\color{blue} Alias} for \$(PV\_PREFIX)\$(M)\_STATUS\_POLL\_DELAY\_MON. In this IOC the macro M="A", by default.
            \end{tabular} \hypertarget{pv:home-enbl-dir}{}\\ \hline
        % --- row ---
        HomeEnblDir-Sel & Enum: None (0), Reverse (1), Forward (2), Both (3) & \begin{tabular}{@{}m{6cm}@{}}
                \textbf{\color{blue} Alias} for \$(PV\_PREFIX)\$(M)\_HOMEALLOWED\_CMD. In this IOC the macro M="A", by default. Enable homing for each direction.
            \end{tabular} \hypertarget{}{}\\ \hline
        % --- row ---
        HomeEnblDir-Sts & Enum: None (0), Reverse (1), Forward (2), Both (3) & \begin{tabular}{@{}m{6cm}@{}}
                \textbf{\color{blue} Alias} for \$(PV\_PREFIX)\$(M)\_HOMEALLOWED\_STATUS. In this IOC the macro M="A", by default. The status of homing enable for each direction.
            \end{tabular} \hypertarget{pv:stop-delay}{}\\ \hline
        % --- row ---
        StopDelay-SP & Float & \begin{tabular}{@{}m{6cm}@{}}
                \textbf{\color{blue} Alias} for \$(PV\_PREFIX)\$(M)\_STOPDELAY\_SP. In this IOC the macro M="A", by default. The delay, in seconds, for the motor to stop after a command.
            \end{tabular} \hypertarget{}{}\\ \hline
        % --- row ---
        StopDelay-RB & Float & \begin{tabular}{@{}m{6cm}@{}}
                \textbf{\color{blue} Alias} for \$(PV\_PREFIX)\$(M)\_STOPDELAY\_MON. In this IOC the macro M="A", by default. The readback value of the stop delay.
            \end{tabular} \hypertarget{pv:amp-gain}{}\\ \hline
        % --- row ---
        AmpGain-Sel & Enum: Zero (0), One (1), Two (2), Three (3) & \begin{tabular}{@{}m{6cm}@{}}
                \textbf{\color{blue} Alias} for \$(PV\_PREFIX)\$(M)\_AMPGAIN\_CMD. In this IOC the macro M="A", by default. This PV controls the internal amplifier gain. The options mean different things for each motor mode:

                Servo motor mode
                \begin{itemize}
                    \item Zero: 0.4 [Amps/Volt]
                    \item One: 0.8 [Amps/Volt]
                    \item Two: 1.6 [Amps/Volt]
                    \item Three: N/A
                \end{itemize}
                Stepper motor mode
                \begin{itemize}
                    \item Zero: 0.75 [Amps per phase]
                    \item One: 1.5 [Amps per phase]
                    \item Two: 3.0 [Amps per phase]
                    \item Three: 6.0 [Amps per phase]
                \end{itemize}
            \end{tabular} \hypertarget{}{}\\ \hline
        % --- row ---
        AmpGain-Sts & Enum: Zero (0), One (1), Two (2), Three (3) & \begin{tabular}{@{}m{6cm}@{}}
                \textbf{\color{blue} Alias} for \$(PV\_PREFIX)\$(M)\_AMPGAIN\_STATUS. In this IOC the macro M="A", by default. This PV displays the internal amplifier gain configuration. The options mean different things for each motor mode:

                Servo motor mode
                \begin{itemize}
                    \item Zero: 0.4 [Amps/Volt]
                    \item One: 0.8 [Amps/Volt]
                    \item Two: 1.6 [Amps/Volt]
                    \item Three: N/A
                \end{itemize}
                Stepper motor mode
                \begin{itemize}
                    \item Zero: 0.75 [Amps per phase]
                    \item One: 1.5 [Amps per phase]
                    \item Two: 3.0 [Amps per phase]
                    \item Three: 6.0 [Amps per phase]
                \end{itemize}
            \end{tabular} \hypertarget{pv:amp-curr-loop-gain}{}\\ \hline
        % --- row ---
        AmpCurrLoopGain-Sel & Enum: Zero (0), Point five (1), One (2), One point five (3), Two (4), Three (5), Four (6) & \begin{tabular}{@{}m{6cm}@{}}
                \textbf{\color{blue} Alias} for \$(PV\_PREFIX)\$(M)\_AMPCLGAIN\_CMD. In this IOC the macro M="A", by default. Current loop gain option for Galil sine drives internal amplifiers. The gain options should be used to compensate for motor inductance. For recommended values, a Galil controller manual should be consulted.
            \end{tabular} \hypertarget{}{}\\ \hline
        % --- row ---
        AmpCurrLoopGain-Sts & Enum: Zero (0), Point five (1), One (2), One point five (3), Two (4), Three (5), Four (6) & \begin{tabular}{@{}m{6cm}@{}}
                \textbf{\color{blue} Alias} for \$(PV\_PREFIX)\$(M)\_AMPCLGAIN\_STATUS. In this IOC the macro M="A", by default. The status of the current loop gain of the sine drive internal amplifier.
            \end{tabular} \hypertarget{pv:low-curr-mode}{}\\ \hline
        % --- row ---
        LowCurrMode-SP & Float & \begin{tabular}{@{}m{6cm}@{}}
                \textbf{\color{blue} Alias} for \$(PV\_PREFIX)\$(M)\_AMPLC\_SP. In this IOC the macro M="A", by default. Configure the low current mode for stepper motors.
                \begin{itemize}
                    \item 0: Drive provides 100\% of holding current when Stepper is at rest.
                    \item 1: Drive drops to 25\% of holding current when Stepper stops moving.
                    \item 2 - 32767: Drive drops to 25\% of holding current after \emph{n} controller \emph{time samples} since the Stepper has stopped.
                \end{itemize}
            \end{tabular} \hypertarget{}{}\\ \hline
        % --- row ---
        LowCurrMode-RB & Float & \begin{tabular}{@{}m{6cm}@{}}
                \textbf{\color{blue} Alias} for \$(PV\_PREFIX)\$(M)\_AMPLC\_MON. In this IOC the macro M="A", by default. Readback value of the low current mode.
                \begin{itemize}
                    \item 0: Drive provides 100\% of holding current when Stepper is at rest.
                    \item 1: Drive drops to 25\% of holding current when Stepper stops moving.
                    \item 2 - 32767: Drive drops to 25\% of holding current after \emph{n} controller \emph{time samples} since the Stepper has stopped.
                \end{itemize}
            \end{tabular} \hypertarget{pv:pos-err-lim}{}\\ \hline
        % --- row ---
        PosErrLim-SP & Float & \begin{tabular}{@{}m{6cm}@{}}
                \textbf{\color{blue} Alias} for \$(PV\_PREFIX)\$(M)\_ERRLIMIT\_SP. In this IOC the macro M="A", by default. The position error limit. When the position error is greater than the specified maximum value and \emph{ErrLimType-Sts} is equal to \emph{Position error} or \emph{Both}, the controller enters an error state and motor power is disabled.
            \end{tabular} \hypertarget{}{}\\ \hline
        % --- row ---
        PosErrLim-RB & Float & \begin{tabular}{@{}m{6cm}@{}}
                \textbf{\color{blue} Alias} for \$(PV\_PREFIX)\$(M)\_ERRLIMIT\_MON. In this IOC the macro M="A", by default. Readback value of the position error limit.
            \end{tabular} \hypertarget{pv:pos-err-mon}{}\\ \hline
        % --- row ---
        PosErr-Mon & Float & \begin{tabular}{@{}m{6cm}@{}}
                \textbf{\color{blue} Alias} for \$(PV\_PREFIX)\$(M)\_ERR\_MON. In this IOC the macro M="A", by default. This PV indicates the difference between commanded position when encoder reading (or hardware step counter).
            \end{tabular} \hypertarget{pv:velo-raw-mon}{}\\ \hline
        % --- row ---
        VeloRaw-Mon & Float & \begin{tabular}{@{}m{6cm}@{}}
                \textbf{\color{blue} Alias} for \$(PV\_PREFIX)\$(M)\_VELOCITYRAW\_MON. In this IOC the macro M="A", by default. This PV indicates the motor velocity in encoder counts per second.
            \end{tabular} \hypertarget{pv:velo-egu-mon}{}\\ \hline
        % --- row ---
        VeloEgu-Mon & Float & \begin{tabular}{@{}m{6cm}@{}}
                \textbf{\color{blue} Alias} for \$(PV\_PREFIX)\$(M)\_VELOCITYEGU\_MON. In this IOC the macro M="A", by default. This PV indicates the motor velocity in engineering units per second (depends on configured encoder resolution).
            \end{tabular} \hypertarget{pv:err-lim-type}{}\\ \hline
        % --- row ---
        ErrLimType-Sel & Enum: Off (0), Position error (1), Limits (2), Both (3) & \begin{tabular}{@{}m{6cm}@{}}
                \textbf{\color{blue} Alias} for \$(PV\_PREFIX)\$(M)\_OFFONERR\_CMD. In this IOC the macro M="A", by default. Configure a controller error associated to a position error, defined by \emph{PosErrLim-RB}, and/or limit switch actuation.
            \end{tabular} \hypertarget{}{}\\ \hline
        % --- row ---
        ErrLimType-Sts & Enum: Off (0), Position error (1), Limits (2), Both (3) & \begin{tabular}{@{}m{6cm}@{}}
                \textbf{\color{blue} Alias} for \$(PV\_PREFIX)\$(M)\_OFFONERR\_STATUS. In this IOC the macro M="A", by default. The configuration status of the limit error type.
            \end{tabular} \hypertarget{pv:axis-mon}{}\\ \hline
        % --- row ---
        Axis-Mon & Enum: A (0), B (1), C (2), D (3), E (4), F (5), G (6), H (7) & \begin{tabular}{@{}m{6cm}@{}}
                \textbf{\color{blue} Alias} for \$(PV\_PREFIX)\$(M)\_AXIS\_STATUS. In this IOC the macro M="A", by default. Indicates the active axis.
            \end{tabular} \hypertarget{pv:mtr-type}{}\\ \hline
        % --- row ---
        MtrType-Sel & Enum: Servo (0), Rev Servo (1), HA Stepper (2), LA Stepper (3), Rev HA Stepper (4), Rev LA Stepper (5), PWM servo (6), PWM rev servo (7), EtherCat Position (8), EtherCat Torque (9), EtherCat Rev Torque (10) & \begin{tabular}{@{}m{6cm}@{}}
                \textbf{\color{blue} Alias} for \$(PV\_PREFIX)\$(M)\_MTRTYPE\_CMD. In this IOC the macro M="A", by default. Configure the motor type. Amplifier must be Off in order for this parameter to be changed.
            \end{tabular} \hypertarget{}{}\\ \hline
        % --- row ---
        MtrType-Sts & Enum: Servo (0), Rev Servo (1), HA Stepper (2), LA Stepper (3), Rev HA Stepper (4), Rev LA Stepper (5), PWM servo (6), PWM rev servo (7), EtherCat Position (8), EtherCat Torque (9), EtherCat Rev Torque (10) & \begin{tabular}{@{}m{6cm}@{}}
                \textbf{\color{blue} Alias} for \$(PV\_PREFIX)\$(M)\_MTRTYPE\_STATUS. In this IOC the macro M="A", by default. The status of motor type configuration.
            \end{tabular} \hypertarget{pv:main-enc-type}{}\\ \hline
        % --- row ---
        MainEncType-Sel & Enum: Normal Quadrature (0), Pulse and Dir (1), Reverse Quadrature (2), Rev Pulse and Dir (3) & \begin{tabular}{@{}m{6cm}@{}}
                \textbf{\color{blue} Alias} for \$(PV\_PREFIX)\$(M)\_MENCTYPE\_CMD. In this IOC the macro M="A", by default. Configure the main encoder type.
            \end{tabular} \hypertarget{}{}\\ \hline
        % --- row ---
        MainEncType-Sts & Enum: Normal Quadrature (0), Pulse and Dir (1), Reverse Quadrature (2), Rev Pulse and Dir (3) & \begin{tabular}{@{}m{6cm}@{}}
                \textbf{\color{blue} Alias} for \$(PV\_PREFIX)\$(M)\_MENCTYPE\_STATUS. In this IOC the macro M="A", by default. Status of the main encoder type configuration.
            \end{tabular} \hypertarget{pv:aux-enc-type}{}\\ \hline
        % --- row ---
        AuxEncType-Sel & Enum: Normal Quadrature (0), Pulse and Dir (1), Reverse Quadrature (2), Rev Pulse and Dir (3) & \begin{tabular}{@{}m{6cm}@{}}
                \textbf{\color{blue} Alias} for \$(PV\_PREFIX)\$(M)\_AENCTYPE\_CMD. In this IOC the macro M="A", by default. Configure the auxiliary encoder type.
            \end{tabular} \hypertarget{}{}\\ \hline
        % --- row ---
        AuxEncType-Sts & Enum: Normal Quadrature (0), Pulse and Dir (1), Reverse Quadrature (2), Rev Pulse and Dir (3) & \begin{tabular}{@{}m{6cm}@{}}
                \textbf{\color{blue} Alias} for \$(PV\_PREFIX)\$(M)\_AENCTYPE\_STATUS. In this IOC the macro M="A", by default. Status of the auxiliary encoder type configuration.
            \end{tabular} \hypertarget{pv:lim-protect-enbl}{}\\ \hline
        % --- row ---
        LimProtectEnbl-Sel & Enum: Off (0), On (1) & \begin{tabular}{@{}m{6cm}@{}}
                \textbf{\color{blue} Alias} for \$(PV\_PREFIX)\$(M)\_WLP\_CMD. In this IOC the macro M="A", by default. Enable Wrong limit protection. This parameter defines if the controller should report an error when the wrong limit switch is activated, i.e., the motor is moving in one direction, but the opposite switch becomes active.
            \end{tabular} \hypertarget{}{}\\ \hline
        % --- row ---
        LimProtectEnbl-Sts & Enum: Off (0), On (1) & \begin{tabular}{@{}m{6cm}@{}}
                \textbf{\color{blue} Alias} for \$(PV\_PREFIX)\$(M)\_WLP\_STATUS. In this IOC the macro M="A", by default. The configuration status of the Wrong limit protection.
            \end{tabular} \hypertarget{pv:lim-protect-mon}{}\\ \hline
        % --- row ---
        LimProtect-Mon & Enum: OK (0), Stopped! (1) & \begin{tabular}{@{}m{6cm}@{}}
                \textbf{\color{blue} Alias} for \$(PV\_PREFIX)\$(M)\_WLPACTIVE\_STATUS. In this IOC the macro M="A", by default. This PV indicates when the \emph{wrong} limit switch has been hit. \emph{Wrong limit protection} must be active for it to happen.
            \end{tabular} \hypertarget{pv:auto-on-off}{}\\ \hline
        % --- row ---
        AutoOnOff-Sel & Enum: Off (0), On (1) & \begin{tabular}{@{}m{6cm}@{}}
                \textbf{\color{blue} Alias} for \$(PV\_PREFIX)\$(M)\_AUTOONOFF\_CMD. In this IOC the macro M="A", by default.
            \end{tabular} \hypertarget{}{}\\ \hline
        % --- row ---
        AutoOnOff-Sts & Enum: Off (0), On (1) & \begin{tabular}{@{}m{6cm}@{}}
                \textbf{\color{blue} Alias} for \$(PV\_PREFIX)\$(M)\_AUTOONOFF\_STATUS. In this IOC the macro M="A", by default.
            \end{tabular} \hypertarget{pv:auto-on-delay}{}\\ \hline
        % --- row ---
        AutoOnDelay-SP & Float & \begin{tabular}{@{}m{6cm}@{}}
                \textbf{\color{blue} Alias} for \$(PV\_PREFIX)\$(M)\_ONDELAY\_SP. In this IOC the macro M="A", by default.
            \end{tabular} \hypertarget{}{}\\ \hline
        % --- row ---
        AutoOnDelay-RB & Float & \begin{tabular}{@{}m{6cm}@{}}
                \textbf{\color{blue} Alias} for \$(PV\_PREFIX)\$(M)\_ONDELAY\_MON. In this IOC the macro M="A", by default.
            \end{tabular} \hypertarget{pv:auto-off-delay}{}\\ \hline
        % --- row ---
        AutoOffDelay-SP & Float & \begin{tabular}{@{}m{6cm}@{}}
                \textbf{\color{blue} Alias} for \$(PV\_PREFIX)\$(M)\_OFFDELAY\_SP. In this IOC the macro M="A", by default.
            \end{tabular} \hypertarget{}{}\\ \hline
        % --- row ---
        AutoOffDelay-RB & Float & \begin{tabular}{@{}m{6cm}@{}}
                \textbf{\color{blue} Alias} for \$(PV\_PREFIX)\$(M)\_OFFDELAY\_MON. In this IOC the macro M="A", by default.
            \end{tabular} \hypertarget{pv:brake-cmd}{}\\ \hline
        % --- row ---
        Brake-Cmd & Enum: Off (0), On (1) & \begin{tabular}{@{}m{6cm}@{}}
                \textbf{\color{blue} Alias} for \$(PV\_PREFIX)\$(M)\_BRAKE\_CMD. In this IOC the macro M="A", by default.
            \end{tabular} \hypertarget{pv:brake-mon}{}\\ \hline
        % --- row ---
        Brake-Mon & Enum: Off (0), On (1) & \begin{tabular}{@{}m{6cm}@{}}
                \textbf{\color{blue} Alias} for \$(PV\_PREFIX)\$(M)\_BRAKE\_STATUS. In this IOC the macro M="A", by default.
            \end{tabular} \hypertarget{pv:auto-brake-enbl}{}\\ \hline
        % --- row ---
        AutoBrakeEnbl-Sel & Enum: Off (0), On (1) & \begin{tabular}{@{}m{6cm}@{}}
                \textbf{\color{blue} Alias} for \$(PV\_PREFIX)\$(M)\_AUTOBRAKE\_CMD. In this IOC the macro M="A", by default.
            \end{tabular} \hypertarget{}{}\\ \hline
        % --- row ---
        AutoBrakeEnbl-Sts & Enum: Off (0), On (1) & \begin{tabular}{@{}m{6cm}@{}}
                \textbf{\color{blue} Alias} for \$(PV\_PREFIX)\$(M)\_AUTOBRAKE\_STATUS. In this IOC the macro M="A", by default.
            \end{tabular} \hypertarget{pv:auto-brake-dig-port}{}\\ \hline
        % --- row ---
        AutoBrakeDigPort-SP & Float & \begin{tabular}{@{}m{6cm}@{}}
                \textbf{\color{blue} Alias} for \$(PV\_PREFIX)\$(M)\_BRAKEPORT\_SP. In this IOC the macro M="A", by default.
            \end{tabular} \hypertarget{}{}\\ \hline
        % --- row ---
        AutoBrakeDigPort-RB & Float & \begin{tabular}{@{}m{6cm}@{}}
                \textbf{\color{blue} Alias} for \$(PV\_PREFIX)\$(M)\_BRAKEPORT\_MON. In this IOC the macro M="A", by default.
            \end{tabular} \hypertarget{pv:auto-brake-delay}{}\\ \hline
        % --- row ---
        AutoBrakeDelay-SP & Float & \begin{tabular}{@{}m{6cm}@{}}
                \textbf{\color{blue} Alias} for \$(PV\_PREFIX)\$(M)\_BRAKEONDELAY\_SP. In this IOC the macro M="A", by default.
            \end{tabular} \hypertarget{}{}\\ \hline
        % --- row ---
        AutoBrakeDelay-RB & Float & \begin{tabular}{@{}m{6cm}@{}}
                \textbf{\color{blue} Alias} for \$(PV\_PREFIX)\$(M)\_BRAKEONDELAY\_MON. In this IOC the macro M="A", by default.
            \end{tabular} \hypertarget{pv:lim-sw-dsbl}{}\\ \hline
        % --- row ---
        LimSwDsbl-Sel & Enum: Off (0), Fwd Disabled (1), Rev Disabled (2), Both Disabled (3) & \begin{tabular}{@{}m{6cm}@{}}
                \textbf{\color{blue} Alias} for \$(PV\_PREFIX)\$(M)\_LIMITDISABLE\_CMD. In this IOC the macro M="A", by default. Disable limit switches.
            \end{tabular} \hypertarget{}{}\\ \hline
        % --- row ---
        LimSwDsbl-Sts & Enum: Off (0), Fwd Disabled (1), Rev Disabled (2), Both Disabled (3) & \begin{tabular}{@{}m{6cm}@{}}
                \textbf{\color{blue} Alias} for \$(PV\_PREFIX)\$(M)\_LIMITDISABLE\_STATUS. In this IOC the macro M="A", by default. Status of limit switch disable.
            \end{tabular} \hypertarget{pv:amp-enbl}{}\\ \hline
        % --- row ---
        AmpEnbl-Sel & Enum: Off (0), On (1) & \begin{tabular}{@{}m{6cm}@{}}
                \textbf{\color{blue} Alias} for \$(PV\_PREFIX)\$(M)\_ON\_CMD. In this IOC the macro M="A", by default. Enable controller amplifier (enable current supply to motor).
            \end{tabular} \hypertarget{}{}\\ \hline
        % --- row ---
        AmpEnbl-Sts & Enum: Off (0), On (1) & \begin{tabular}{@{}m{6cm}@{}}
                \textbf{\color{blue} Alias} for \$(PV\_PREFIX)\$(M)\_ON\_STATUS. In this IOC the macro M="A", by default. Status of controller amplifier.
            \end{tabular} \hypertarget{pv:biss-crc-mon}{}\\ \hline
        % --- row ---
        BiSSCRC-Mon & Enum: Valid CRC (0), Invalid CRC (1) & \begin{tabular}{@{}m{6cm}@{}}
                \textbf{\color{blue} Alias} for \$(PV\_PREFIX)\$(M)\_BISSSTAT\_CRC. Status of the BiSS CRC.
            \end{tabular} \hypertarget{pv:biss-err-mon}{}\\ \hline
        % --- row ---
        BiSSErr-Mon & Enum: No Error (0), Error (1) & \begin{tabular}{@{}m{6cm}@{}}
                \textbf{\color{blue} Alias} for \$(PV\_PREFIX)\$(M)\_BISSSTAT\_ERROR. BiSS communication error status.
            \end{tabular} \hypertarget{pv:biss-poll}{}\\ \hline
        % --- row ---
        BiSSPoll-Sel & Enum: No (0), Yes (1) & \begin{tabular}{@{}m{6cm}@{}}
                \textbf{\color{blue} Alias} for \$(PV\_PREFIX)\$(M)\_BISSSTAT\_POLL\_CMD. BiSS polling configuration.
            \end{tabular} \hypertarget{}{}\\ \hline
        % --- row ---
        BiSSPoll-Sts & No (0), Yes (1) & \begin{tabular}{@{}m{6cm}@{}}
                \textbf{\color{blue} Alias} for \$(PV\_PREFIX)\$(M)\_BISSSTAT\_POLL. Status of BiSS polling.
            \end{tabular} \hypertarget{pv:biss-timeout-mon}{}\\ \hline
        % --- row ---
        BiSSTimeout-Mon & No Timeout (0), Timeout (1) & \begin{tabular}{@{}m{6cm}@{}}
                \textbf{\color{blue} Alias} for \$(PV\_PREFIX)\$(M)\_BISSSTAT\_TIMEOUT. Status of the BiSS timeout.
            \end{tabular} \hypertarget{pv:biss-warn-mon}{}\\ \hline
        % --- row ---
        BiSSWarn-Mon & No Warning (0), Warning (1) & \begin{tabular}{@{}m{6cm}@{}}
                \textbf{\color{blue} Alias} for \$(PV\_PREFIX)\$(M)\_BISSSTAT\_WARN. BiSS warning status.
            \end{tabular} \hypertarget{pv:biss-capable-mon}{}\\ \hline
        % --- row ---
        BiSSCapable-Mon & No (0), Yes (1) & \begin{tabular}{@{}m{6cm}@{}}
                \textbf{\color{blue} Alias} for \$(PV\_PREFIX)BISSCAPABLE\_STATUS. Status indicating if the controller supports BiSS-C communication.
            \end{tabular} \hypertarget{pv:biss-clk-div}{}\\ \hline
        % --- row ---
        BiSSClkDiv-SP & Long & \begin{tabular}{@{}m{6cm}@{}}
                \textbf{\color{blue} Alias} for \$(PV\_PREFIX)\$(M)\_BISSCD\_SP. Configure BiSS clock divider.
            \end{tabular} \hypertarget{}{}\\ \hline
        % --- row ---
        BiSSClkDiv-RB & Long & \begin{tabular}{@{}m{6cm}@{}}
                \textbf{\color{blue} Alias} for \$(PV\_PREFIX)\$(M)\_BISSCD\_MON. Readback value of the BiSS clock divider.
            \end{tabular} \hypertarget{pv:biss-num-bits-1}{}\\ \hline
        % --- row ---
        BiSSNumBits1-SP & Long & \begin{tabular}{@{}m{6cm}@{}}
                \textbf{\color{blue} Alias} for \$(PV\_PREFIX)\$(M)\_BISSDATA1\_SP. Configure the number of data bits for BiSS communication. This is the first parameter used for it. For absolute encoders \emph{BiSSNumBits1-RB} and \emph{BiSSNumBits2-RB} should have the same value. If the absolute position rolls over, \emph{BiSSNumBits1-RB} should be equal to -1 $\times$ \emph{BiSSNumBits2-RB}. This allows for the firmware to compensate, counting past the boundaries. Renishaw encoders transmit a leading zero, so the number of bits should be incremented by 1 when configuring this parameter.
            \end{tabular} \hypertarget{}{}\\ \hline
        % --- row ---
        BiSSNumBits1-RB & Long & \begin{tabular}{@{}m{6cm}@{}}
                \textbf{\color{blue} Alias} for \$(PV\_PREFIX)\$(M)\_BISSDATA1\_MON. Readback value of the \emph{first} parameter which determines the number of data bits used by the BiSS communication.
            \end{tabular} \hypertarget{pv:biss-num-bits-2}{}\\ \hline
        % --- row ---
        BiSSNumBits2-SP & Long & \begin{tabular}{@{}m{6cm}@{}}
                \textbf{\color{blue} Alias} for \$(PV\_PREFIX)\$(M)\_BISSDATA2\_SP. Configure the number of data bits for BiSS communication. This is the second parameter used for it. For absolute encoders \emph{BiSSNumBits1-RB} and \emph{BiSSNumBits2-RB} should have the same value. If the absolute position rolls over, \emph{BiSSNumBits1-RB} should be equal to -1 $\times$ \emph{BiSSNumBits2-RB}. This allows for the firmware to compensate, counting past the boundaries. Renishaw encoders transmit a leading zero, so the number of bits should be incremented by 1 when configuring this parameter.
            \end{tabular} \hypertarget{}{}\\ \hline
        % --- row ---
        BiSSNumBits2-RB & Long & \begin{tabular}{@{}m{6cm}@{}}
                \textbf{\color{blue} Alias} for \$(PV\_PREFIX)\$(M)\_BISSDATA2\_MON. Readback value of the \emph{second} parameter which determines the number of data bits used by the BiSS communication.
            \end{tabular} \hypertarget{pv:biss-num-zero-pad}{}\\ \hline
        % --- row ---
        BiSSNumZeroPad-SP & Long & \begin{tabular}{@{}m{6cm}@{}}
                \textbf{\color{blue} Alias} for \$(PV\_PREFIX)\$(M)\_BISSZP\_SP. Configure the number of zero padding bits required by the BiSS communication.
            \end{tabular} \hypertarget{}{}\\ \hline
        % --- row ---
        BiSSNumZeroPad-RB & Long & \begin{tabular}{@{}m{6cm}@{}}
                \textbf{\color{blue} Alias} for \$(PV\_PREFIX)\$(M)\_BISSZP\_MON. Readback value of configured number of zero padding bits for the BiSS communication.
            \end{tabular} \hypertarget{pv:biss-in}{}\\ \hline
        % --- row ---
        BiSSIn-Sel & Enum: Off (0), Replace main (1), Replace aux (2) & \begin{tabular}{@{}m{6cm}@{}}
                \textbf{\color{blue} Alias} for \$(PV\_PREFIX)\$(M)\_BISSINPUT\_CMD. Configure how the BiSS data input should be used: off, main encoder or auxiliary encoder.
            \end{tabular} \hypertarget{}{}\\ \hline
        % --- row ---
        BiSSIn-Sts & Enum: Off (0), Replace main (1), Replace aux (2) & \begin{tabular}{@{}m{6cm}@{}}
                \textbf{\color{blue} Alias} for \$(PV\_PREFIX)\$(M)\_BISSINPUT\_STATUS. The configuration status of the BiSS input.
            \end{tabular} \hypertarget{pv:biss-lvl}{}\\ \hline
        % --- row ---
        BiSSLvl-Sel & Enum: Low/Low (0), Low/High (1), High/Low (2), High/High (3) & \begin{tabular}{@{}m{6cm}@{}}
                \textbf{\color{blue} Alias} for \$(PV\_PREFIX)\$(M)\_BISSLEVEL\_CMD. Configure BiSS level.
            \end{tabular} \hypertarget{}{}\\ \hline
        % --- row ---
        BiSSLvl-Sts & Enum: Low/Low (0), Low/High (1), High/Low (2), High/High (3) & \begin{tabular}{@{}m{6cm}@{}}
                \textbf{\color{blue} Alias} for \$(PV\_PREFIX)\$(M)\_BISSLEVEL\_STATUS. Status of BiSS level configuration.
            \end{tabular} \hypertarget{pv:dig-out-0}{}\\ \hline
        % --- row ---
        DigOut0-Sel & Enum: Off (0), On (1) & \begin{tabular}{@{}m{6cm}@{}}
                \textbf{\color{blue} Alias} for \$(PV\_PREFIX)\$(DIG\_OUT0)\_CMD. Configure digital output 0 state.
            \end{tabular} \hypertarget{}{}\\ \hline
        % --- row ---
        DigOut0-Sts & Enum: Off (0), On (1) & \begin{tabular}{@{}m{6cm}@{}}
                \textbf{\color{blue} Alias} for \$(PV\_PREFIX)\$(DIG\_OUT0)\_STATUS. Status of digital output 0.
            \end{tabular} \hypertarget{pv:dig-out-1}{}\\ \hline
        % --- row ---
        DigOut1-Sel & Enum: Off (0), On (1) & \begin{tabular}{@{}m{6cm}@{}}
                \textbf{\color{blue} Alias} for \$(PV\_PREFIX)\$(DIG\_OUT1)\_CMD. Configure digital output 1 state.
            \end{tabular} \hypertarget{}{}\\ \hline
        % --- row ---
        DigOut1-Sts & Enum: Off (0), On (1) & \begin{tabular}{@{}m{6cm}@{}}
                \textbf{\color{blue} Alias} for \$(PV\_PREFIX)\$(DIG\_OUT1)\_STATUS. Status digital output 1.
            \end{tabular} \hypertarget{pv:dig-out-2}{}\\ \hline
        % --- row ---
        DigOut2-Sel & Enum: Off (0), On (1) & \begin{tabular}{@{}m{6cm}@{}}
                \textbf{\color{blue} Alias} for \$(PV\_PREFIX)\$(DIG\_OUT2)\_CMD. Configure digital output 2 state.
            \end{tabular} \hypertarget{}{}\\ \hline
        % --- row ---
        DigOut2-Sts & Enum: Off (0), On (1) & \begin{tabular}{@{}m{6cm}@{}}
                \textbf{\color{blue} Alias} for \$(PV\_PREFIX)\$(DIG\_OUT2)\_STATUS. Status of digital output 2.
            \end{tabular} \hypertarget{pv:dig-out-3}{}\\ \hline
        % --- row ---
        DigOut3-Sel & Enum: Off (0), On (1) & \begin{tabular}{@{}m{6cm}@{}}
                \textbf{\color{blue} Alias} for \$(PV\_PREFIX)\$(DIG\_OUT3)\_CMD. Configure digital output 3 state.
            \end{tabular} \hypertarget{}{}\\ \hline
        % --- row ---
        DigOut3-Sts & Enum: Off (0), On (1) & \begin{tabular}{@{}m{6cm}@{}}
                \textbf{\color{blue} Alias} for \$(PV\_PREFIX)\$(DIG\_OUT3)\_STATUS. Status of digital output 3.
            \end{tabular} \hypertarget{pv:dig-in-0}{}\\ \hline
        % --- row ---
        DigIn0-Mon & Enum: Off (0), On (1) & \begin{tabular}{@{}m{6cm}@{}}
                \textbf{\color{blue} Alias} for \$(PV\_PREFIX)\$(DIG\_IN0)\_STATUS. Status of digital input 0.
            \end{tabular} \hypertarget{pv:dig-in-1}{}\\ \hline
        % --- row ---
        DigIn1-Mon & Enum: Off (0), On (1) & \begin{tabular}{@{}m{6cm}@{}}
                \textbf{\color{blue} Alias} for \$(PV\_PREFIX)\$(DIG\_IN1)\_STATUS. Status of digital input 1.
            \end{tabular} \hypertarget{pv:dig-in-2}{}\\ \hline
        % --- row ---
        DigIn2-Mon & Enum: Off (0), On (1) & \begin{tabular}{@{}m{6cm}@{}}
                \textbf{\color{blue} Alias} for \$(PV\_PREFIX)\$(DIG\_IN2)\_STATUS. Status of digital input 2.
            \end{tabular} \hypertarget{pv:dig-in-3}{}\\ \hline
        % --- row ---
        DigIn3-Mon & Enum: Off (0), On (1) & \begin{tabular}{@{}m{6cm}@{}}
                \textbf{\color{blue} Alias} for \$(PV\_PREFIX)\$(DIG\_IN3)\_STATUS. Status of digital input 3.
            \end{tabular} \hypertarget{pv:dig-in-4}{}\\ \hline
        % --- row ---
        DigIn4-Mon & Enum: Off (0), On (1) & \begin{tabular}{@{}m{6cm}@{}}
                \textbf{\color{blue} Alias} for \$(PV\_PREFIX)\$(DIG\_IN4)\_STATUS. Status of digital input 4.
            \end{tabular} \hypertarget{pv:dig-in-5}{}\\ \hline
        % --- row ---
        DigIn5-Mon & Enum: Off (0), On (1) & \begin{tabular}{@{}m{6cm}@{}}
                \textbf{\color{blue} Alias} for \$(PV\_PREFIX)\$(DIG\_IN5)\_STATUS. Status of digital input 5.
            \end{tabular} \hypertarget{pv:dig-in-6}{}\\ \hline
        % --- row ---
        DigIn6-Mon & Enum: Off (0), On (1) & \begin{tabular}{@{}m{6cm}@{}}
                \textbf{\color{blue} Alias} for \$(PV\_PREFIX)\$(DIG\_IN6)\_STATUS. Status of digital input 6.
            \end{tabular} \hypertarget{pv:dig-in-7}{}\\ \hline
        % --- row ---
        DigIn7-Mon & Enum: Off (0), On (1) & \begin{tabular}{@{}m{6cm}@{}}
                \textbf{\color{blue} Alias} for \$(PV\_PREFIX)\$(DIG\_IN7)\_STATUS. Status of digital input 7.
            \end{tabular} \hypertarget{pv:analog-out-0}{}\\ \hline
        % --- row ---
        AnalogOut0-SP & Float & \begin{tabular}{@{}m{6cm}@{}}
                \textbf{\color{blue} Alias} for \$(PV\_PREFIX)\$(ANALOG\_OUT0)\_SP. Configure analog output 0 level, in volts. For the DMC 30017, this analog output is only available when the controller is configured for brushless motor control.
            \end{tabular} \hypertarget{}{}\\ \hline
        % --- row ---
        AnalogOut0-RB & Float & \begin{tabular}{@{}m{6cm}@{}}
                \textbf{\color{blue} Alias} for \$(PV\_PREFIX)\$(ANALOG\_OUT0)\_MON. Readback value of analog output 0 level, in volts. For the DMC 30017, this analog output is only available when the controller is configured for brushless motor control.
            \end{tabular} \hypertarget{pv:analog-out-0-dead-band}{}\\ \hline
        % --- row ---
        AnalogOut0DeadBand-SP & Float & \begin{tabular}{@{}m{6cm}@{}}
                \textbf{\color{blue} Alias} for \$(PV\_PREFIX)\$(ANALOG\_OUT0)Deadb\_SP. Configure analog output 0 deadband, in volts. For the DMC 30017, this analog output is only available when the controller is configured for brushless motor control.
            \end{tabular} \hypertarget{}{}\\ \hline
        % --- row ---
        AnalogOut0DeadBand-RB & Float & \begin{tabular}{@{}m{6cm}@{}}
                \textbf{\color{blue} Alias} for \$(PV\_PREFIX)\$(ANALOG\_OUT0)Deadb\_SP. Readback value of the analog output 0 deadband, in volts. For the DMC 30017, this analog output is only available when the controller is configured for brushless motor control.
            \end{tabular} \hypertarget{pv:analog-out-1}{}\\ \hline
        % --- row ---
        AnalogOut1-SP & Float & \begin{tabular}{@{}m{6cm}@{}}
                \textbf{\color{blue} Alias} for \$(PV\_PREFIX)\$(ANALOG\_OUT1)\_SP. Configure analog output 1 level, in volts.
            \end{tabular} \hypertarget{}{}\\ \hline
        % --- row ---
        AnalogOut1-RB & Float & \begin{tabular}{@{}m{6cm}@{}}
                \textbf{\color{blue} Alias} for \$(PV\_PREFIX)\$(ANALOG\_OUT1)\_MON. Readback value of analog output 1 level, in volts.
            \end{tabular} \hypertarget{pv:analog-out-1-dead-band}{}\\ \hline
        % --- row ---
        AnalogOut1DeadBand-SP & Float & \begin{tabular}{@{}m{6cm}@{}}
                \textbf{\color{blue} Alias} for \$(PV\_PREFIX)\$(ANALOG\_OUT1)Deadb\_SP. Configure analog output 1 deadband, in volts.
            \end{tabular} \hypertarget{}{}\\ \hline
        % --- row ---
        AnalogOut1DeadBand-RB & Float & \begin{tabular}{@{}m{6cm}@{}}
                \textbf{\color{blue} Alias} for \$(PV\_PREFIX)\$(ANALOG\_OUT1)Deadb\_SP. Readback value of the analog output 1 deadband, in volts.
            \end{tabular} \hypertarget{pv:analog-in-0-mon}{}\\ \hline
        % --- row ---
        AnalogIn0-Mon & Float & \begin{tabular}{@{}m{6cm}@{}}
                \textbf{\color{blue} Alias} for \$(PV\_PREFIX)\$(ANALOG\_IN0)\_MON. Display the level, in volts, of the analog input 0.
            \end{tabular} \hypertarget{pv:analog-in-0-dead-band}{}\\ \hline
        % --- row ---
        AnalogIn0DeadBand-SP & Float & \begin{tabular}{@{}m{6cm}@{}}
                \textbf{\color{blue} Alias} for \$(PV\_PREFIX)\$(ANALOG\_IN0)Deadb\_SP. Configure the analog input 0 deadband, in volts.
            \end{tabular} \hypertarget{}{}\\ \hline
        % --- row ---
        AnalogIn0DeadBand-RB & Float & \begin{tabular}{@{}m{6cm}@{}}
                \textbf{\color{blue} Alias} for \$(PV\_PREFIX)\$(ANALOG\_IN0)Deadb\_SP. Readback value of the configured deadband for the analog input 0, in volts.
            \end{tabular} \hypertarget{pv:analog-in-1-mon}{}\\ \hline
        % --- row ---
        AnalogIn1-Mon & Float & \begin{tabular}{@{}m{6cm}@{}}
                \textbf{\color{blue} Alias} for \$(PV\_PREFIX)\$(ANALOG\_IN1)\_MON. Display the level, in volts, of the analog input 1.
            \end{tabular} \hypertarget{pv:analog-in-1-dead-band}{}\\ \hline
        % --- row ---
        AnalogIn1DeadBand-SP & Float & \begin{tabular}{@{}m{6cm}@{}}
                \textbf{\color{blue} Alias} for \$(PV\_PREFIX)\$(ANALOG\_IN1)Deadb\_SP. Configure the analog input 1 deadband, in volts.
            \end{tabular} \hypertarget{}{}\\ \hline
        % --- row ---
        AnalogIn1DeadBand-RB & Float & \begin{tabular}{@{}m{6cm}@{}}
                \textbf{\color{blue} Alias} for \$(PV\_PREFIX)\$(ANALOG\_IN1)Deadb\_SP. Readback value of the configured deadband for the analog input 1, in volts.
            \end{tabular} \hypertarget{pv:pos-diff-mon}{}\\ \hline
        % --- row ---
        PosDiff-Mon & Float & \begin{tabular}{@{}m{6cm}@{}}
                Difference between encoder reading and commanded position, in engineering units. This PV reads its value from the \emph{Mtr.DIFF} PV.
            \end{tabular} \hypertarget{pv:done-mov-mon}{}\\ \hline
        % --- row ---
        DoneMov-Mon & Enum: No (0), Yes (1) & \begin{tabular}{@{}m{6cm}@{}}
                Indicates when the motor has completely finished the movement to a given position. If the motor fails to reach the position, but has exhausted all retries, the motion is considered to be finished. This PV reads its value from the value the \emph{Mtr.DMOV} PV.
            \end{tabular} \hypertarget{pv:rst-cmd}{}\\ \hline
        % --- row ---
        Rst-Cmd & Enum: Off (0), On (1) & \begin{tabular}{@{}m{6cm}@{}}
                Reset command. When set to 1, this PV resets the controller. \textcolor{red}{No command should be sent to the controller while it is resetting}. The reset operation status is displayed by the \hyperlink{pv:rst-done-mon}{RstDone-Mon} PV.
            \end{tabular} \hypertarget{pv:rst-done-mon}{}\\ \hline
        % --- row ---
        RstDone-Mon & Enum: In Progress (0), Finished (1) & \begin{tabular}{@{}m{6cm}@{}}
                Status of the reset operation. It indicates whether there is an ongoing reset operation or not.
            \end{tabular} \hypertarget{pv:high-lim-sw-mon}{}\\ \hline
        % --- row ---
        HighLimSw-Mon & Enum: No (0), Yes (1) & \begin{tabular}{@{}m{6cm}@{}}
                High limit switch active status. It indicates whether the high (forward) limit switch is active (is being pressed). The \hyperlink{limit switch type}{pv:lim-sw-type} PV defines whether the switch is normally open or normally closed.
            \end{tabular} \hypertarget{pv:low-lim-sw-mon}{}\\ \hline
        % --- row ---
        LowLimSw-Mon & Enum: No (0), Yes (1) & \begin{tabular}{@{}m{6cm}@{}}
                Low limit switch active status. It indicates whether the low (reverse) limit switch is active (is being pressed). The \hyperlink{limit switch type}{pv:lim-sw-type} PV defines whether the switch is normally open or normally closed.
            \end{tabular} \hypertarget{pv:enc-pos-mon}{}\\ \hline
        % --- row ---
        EncPos-Mon & Float & \begin{tabular}{@{}m{6cm}@{}}
                Encoder position reading. When the encoder use is enabled, this PV displays the current encoder reading, in engineering units, based on its configured per step resolution.
            \end{tabular} \hypertarget{pv:move-relative-cmd}{}\\ \hline
        % --- row ---
        MoveRelative-Cmd & Float & \begin{tabular}{@{}m{6cm}@{}}
                A command which moves the motor by the specified amount, in engineering units, relative to its current position.
            \end{tabular} \hypertarget{pv:stop-cmd}{}\\ \hline
        % --- row ---
        Stop-Cmd & Enum: Off (0), On (1) & \begin{tabular}{@{}m{6cm}@{}}
                Command to stop the motor. When this PV is set to 1, the motor decelerates to a stop according to the acceleration/deceleration time specified by Mtr.ACCL.
            \end{tabular} \hypertarget{pv:tweak-fwd-cmd}{}\\ \hline
        % --- row ---
        TweakFwd-Cmd & Enum: Off (0), On (1) & \begin{tabular}{@{}m{6cm}@{}}
                Command motor to move in the forward direction (depends on Mtr.DIR configuration) by the distance specified by \hyperlink{pv:tweak-step}{TweakStep-RB} PV. Setting the PV to 0 does NOT generate a command.
            \end{tabular} \hypertarget{pv:tweak-back-cmd}{}\\ \hline
        % --- row ---
        TweakBack-Cmd & Enum: Off (0), On (1) & \begin{tabular}{@{}m{6cm}@{}}
                Command motor to move in the backward direction (depends on Mtr.DIR configuration) by the distance specified by \hyperlink{pv:tweak-step}{TweakStep-RB} PV. Setting the PV to 0 does NOT generate a command.
            \end{tabular} \hypertarget{pv:tweak-step}{}\\ \hline
        % --- row ---
        TweakStep-SP & Float & \begin{tabular}{@{}m{6cm}@{}}
                The step, in engineering units, by which the motor is moved when a tweak command is sent.
            \end{tabular} \hypertarget{}{}\\ \hline
        % --- row ---
        TweakStep-RB & Float & \begin{tabular}{@{}m{6cm}@{}}
                The readback value of the tweak step.
            \end{tabular} \hypertarget{pv:abs-pos}{}\\ \hline
        % --- row ---
        AbsPos-SP & Float & \begin{tabular}{@{}m{6cm}@{}}
                Command the motor to the specified absolute position, in engineering units.
            \end{tabular} \hypertarget{}{}\\ \hline
        % --- row ---
        AbsPos-RB & Float & \begin{tabular}{@{}m{6cm}@{}}
                The readback value of the absolute position, in engineering units. This is the value to which the hardware is commanded to move to, but it does not reflect the encoder reading, when one is being used.
            \end{tabular} \hypertarget{pv:mtr-on-macro-cte}{}\\ \hline
        % --- row ---
        MtrOnMacro-Cte & String (char[40]) & \begin{tabular}{@{}m{6cm}@{}}
                \emph{Motor On} macro value passed at initialization. When this macro is set, its value overrides autosave at initialization.
            \end{tabular} \hypertarget{pv:mtr-type-macro-cte}{}\\ \hline
        % --- row ---
        MtrTypeMacro-Cte & String (char[40]) & \begin{tabular}{@{}m{6cm}@{}}
                \emph{Motor Type} macro value passed at initialization. When this macro is set, its value overrides autosave at initialization.
            \end{tabular} \hypertarget{pv:amp-gain-macro-cte}{}\\ \hline
        % --- row ---
        AmpGainMacro-Cte & String (char[40]) & \begin{tabular}{@{}m{6cm}@{}}
                \emph{Amplifier Gain} macro value passed at initialization. When this macro is set, its value overrides autosave at initialization.
            \end{tabular} \hypertarget{pv:dir-macro-cte}{}\\ \hline
        % --- row ---
        DirMacro-Cte & String (char[40]) & \begin{tabular}{@{}m{6cm}@{}}
                \emph{Motor Direction} macro value passed at initialization. When this macro is set, its value overrides autosave at initialization.
            \end{tabular} \hypertarget{pv:enc-type-macro-cte}{}\\ \hline
        % --- row ---
        EncTypeMacro-Cte & String (char[40]) & \begin{tabular}{@{}m{6cm}@{}}
                \emph{Encoder Type} macro value passed at initialization. When this macro is set, its value overrides autosave at initialization.
            \end{tabular} \hypertarget{pv:biss-poll-macro-cte}{}\\ \hline
        % --- row ---
        BiSSPollMacro-Cte & String (char[40]) & \begin{tabular}{@{}m{6cm}@{}}
                \emph{BiSS Polling Enable} macro value passed at initialization. When this macro is set, its value overrides autosave at initialization.
            \end{tabular} \hypertarget{pv:biss-clk-div-macro-cte}{}\\ \hline
        % --- row ---
        BiSSClkDivMacro-Cte & String (char[40]) & \begin{tabular}{@{}m{6cm}@{}}
                \emph{BiSS Clock Divider} macro value passed at initialization. When this macro is set, its value overrides autosave at initialization.
            \end{tabular} \hypertarget{pv:biss-data-1--macro-cte}{}\\ \hline
        % --- row ---
        BiSSData1Macro-Cte & String (char[40]) & \begin{tabular}{@{}m{6cm}@{}}
                \emph{BiSS Data 1} macro value passed at initialization. When this macro is set, its value overrides autosave at initialization.
            \end{tabular} \hypertarget{pv:biss-data-2-macro-cte}{}\\ \hline
        % --- row ---
        BiSSData2Macro-Cte & String (char[40]) & \begin{tabular}{@{}m{6cm}@{}}
                \emph{BiSS Data 2} macro value passed at initialization. When this macro is set, its value overrides autosave at initialization.
            \end{tabular} \hypertarget{pv:biss-zero-pad-macro-cte}{}\\ \hline
        % --- row ---
        BiSSZeroPadMacro-Cte & String (char[40]) & \begin{tabular}{@{}m{6cm}@{}}
                \emph{BiSS Zero Padding} macro value passed at initialization. When this macro is set, its value overrides autosave at initialization.
            \end{tabular} \hypertarget{pv:biss-in-macro-cte}{}\\ \hline
        % --- row ---
        BiSSInMacro-Cte & String (char[40]) & \begin{tabular}{@{}m{6cm}@{}}
                \emph{BiSS Input} macro value passed at initialization. When this macro is set, its value overrides autosave at initialization.
            \end{tabular} \hypertarget{pv:biss-lvl-macro-cte}{}\\ \hline
        % --- row ---
        BiSSLvlMacro-Cte & String (char[40]) & \begin{tabular}{@{}m{6cm}@{}}
                \emph{BiSS Level} macro value passed at initialization. When this macro is set, its value overrides autosave at initialization\cite{mtrrec}.
            \end{tabular} \hypertarget{}{}\\ \hline
    \end{longtable}

\end{document}
\grid

